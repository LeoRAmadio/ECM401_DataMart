% PREÂMBULO DO DOCUMENTO

\documentclass[
    12pt,            % Tamanho da fonte
    a4paper,         % Tamanho do papel
    oneside,         % Impressão em um lado
    brazil           % Idioma para hifenização
]{abntex2}

% --- PACOTES BÁSICOS ---

\usepackage[utf8]{inputenc}     % Codificação de caracteres
\usepackage[T1]{fontenc}        % Codificação da fonte
\usepackage[brazil]{babel}      % Tradução de termos para o português
\usepackage{graphicx}           % Para incluir imagens
\usepackage{hyperref}           % Pacote principal para links
\hypersetup{hidelinks}          % Para links clicáveis (sem caixas coloridas)
\usepackage{placeins}           % Para barreiras no posicionamento das imagens

% --- PACOTES DE FORMATAÇÃO ABNT (TCC) ---
% Define as margens corretas: 3cm superior/esquerda, 2cm inferior/direita
\usepackage[top=3cm, bottom=2cm, left=3cm, right=2cm]{geometry}

% --- Define a fonte como Times New Roman (Padrão ABNT) ---
\usepackage{times}

% --- Adicionado pacote para tabelas ---
\usepackage{booktabs}
\usepackage{longtable}          % Para tabelas que podem quebrar a página
\usepackage{array}              % Para definir largura de colunas

% --- Pacote para código ---
\usepackage{listings}
\lstset{
  basicstyle=\ttfamily\small,
  keywordstyle=\color{blue},
  stringstyle=\color{red},
  commentstyle=\color{gray},
  breaklines=true,
  showstringspaces=false
}


% --- PACOTES PARA BIBTEX ABNT (Estilo Numérico) ---
\usepackage[num]{abntex2cite}   

% --- INFORMAÇÕES DO TRABALHO E AUTORES ---
\titulo{Plataforma de BI para Engenharia de Confiabilidade - Análise de Degradação de Motores Turbofan (C-MAPSS)} % Título atualizado
\autor{%
    André Solano Ferreira Rodrigues Maiolini (19.02012-0) \\
    Durval Consorti Soranz de Barros Santos (22.01097-0) \\
    Leonardo Roberto Amadio (22.01300-8) \\
    Lucas Castanho Paganotto Carvalho (22.00921-3)
}

\instituicao{%
  Instituto Mauá de Tecnologia
  \par
  Engenharia de Computação 
}

% --- Informações adicionais da folha de rosto ---
\preambulo{%
    Trabalho apresentado para aprovação na disciplina ECM401 - Banco de Dados, do Instituto Mauá de Tecnologia.
    \par \vspace{0.5cm}
    Professor: Dr. Antônio Fernando Nunes Guardado
}

\local{São Caetano Sul -- SP}
\data{2025}

% --- CONFIGURAÇÕES DE APARÊNCIA ABNT ---
\renewcommand{\baselinestretch}{1.5}

% --- INÍCIO DO DOCUMENTO ---
\begin{document}

% --- ELEMENTOS PRÉ-TEXTUAIS ---
\imprimircapa
\imprimirfolhaderosto

% RESUMO
\begin{resumo}
     Este projeto propõe o desenvolvimento de uma plataforma de Business Intelligence voltada à Engenharia de Confiabilidade, utilizando o dataset C-MAPSS da NASA. O objetivo é demonstrar como dados de sensores de motores turbofan — organizados como séries temporais — podem ser transformados em indicadores que auxiliem engenheiros e gestores na avaliação de degradação, desempenho e vida útil de componentes. O foco do BI (\textit{Business Intelligence}) não é prever falhas via aprendizado de máquina, mas monitorar, contextualizar e analisar o comportamento dos motores em diferentes cenários de teste, oferecendo uma base quantitativa para decisões de engenharia, como ajustes de projeto e definição de políticas de manutenção.
     \vspace{\onelineskip}
     \noindent
     \newline \\ \textbf{Palavras-chave}: Business Intelligence. Engenharia de Confiabilidade. DataMart. C-MAPSS. Manutenção Preditiva. Aeronáutica.
\end{resumo}

% --- SUMÁRIO ---
\tableofcontents* \newpage

% --- ELEMENTOS TEXTUAIS ---
\textual

% CAPÍTULO 1: INTRODUÇÃO
\chapter{Introdução}

Este projeto propõe o desenvolvimento de uma plataforma de Business Intelligence (BI) voltada à Engenharia de Confiabilidade, utilizando o dataset C-MAPSS da NASA. O objetivo é demonstrar como dados de sensores de motores turbofan — organizados como séries temporais — podem ser transformados em indicadores que auxiliem engenheiros e gestores na avaliação de degradação, desempenho e vida útil de componentes.
O foco do BI não é prever falhas via aprendizado de máquina, mas monitorar, contextualizar e analisar o comportamento dos motores em diferentes cenários de teste, oferecendo uma base quantitativa para decisões de engenharia, como ajustes de projeto e definição de políticas de manutenção.

\section{IoT, Big Data e a Evolução da Manutenção}

A indústria aeronáutica moderna é uma das maiores geradoras de dados do mundo. Uma única aeronave comercial pode gerar terabytes de dados por voo, coletados por milhares de sensores embarcados em uma arquitetura complexa de Internet das Coisas (IoT) \cite{adamopoulou2023applications, badea2018big}. Esses dados representam um desafio de Big Data, mas também uma oportunidade sem precedentes para otimizar operações.

Um exemplo prático e ilustrativo do volume de dados gerados no contexto aeronáutico pode ser observado em uma aeronave Boeing 747 \cite{badea2018big}. Essa aeronave possui dois motores, sendo que cada um deles é capaz de gerar aproximadamente 20 terabytes de informações por hora de voo. Conforme ilustrado na Figura \ref{fig:terabytes}, em um trajeto de seis horas, estima-se a produção de cerca de 240 terabytes de dados por aeronave. Quando esse valor é extrapolado para a média de 28.537 voos comerciais diários realizados nos Estados Unidos, o resultado é um volume anual superior a 2,4 bilhões de terabytes. Esse cenário evidencia o enorme potencial das aeronaves comerciais como fontes de dados em larga escala, reforçando a importância do uso de técnicas de Big Data e Internet das Coisas (IoT) para o processamento, análise e aproveitamento dessas informações.

\FloatBarrier

\begin{figure}[htbp]
  \centering
  \includegraphics[width=\textwidth]{images/terabytes_iot_aircraft.png}
  \caption{Dados de sensores em voos entre Nova Iorque e Los Angeles.}
  \label{fig:terabytes}
\end{figure}

\FloatBarrier

Historicamente, a manutenção aeronáutica era dominada pela abordagem conservadora baseada em tempo (TBM - \textit{Time-Based Maintenance}), onde componentes são substituídos após um número fixo de horas de voo, independentemente de sua condição real. Esta abordagem, embora segura, é extremamente cara e ineficiente - significando a troca de peças que estariam ainda adequadas para o uso.

A disponibilidade desses vastos conjuntos de dados impulsiona a transição para a Manutenção Baseada na Condição (CBM - \textit{Condition-Based Maintenance}) e para a Manutenção Preditiva (PdM). O BI é a ferramenta-chave que permite às equipes de engenharia consumir esse volume de dados e identificar os "sintomas digitais" \ de degradação antes que uma falha ocorra.

\section{Objetivos}

\subsection{Objetivo Geral}
Construir um DataMart e um conjunto de dashboards de BI que consolidem dados de séries temporais do C-MAPSS em indicadores de confiabilidade, permitindo análises comparativas de degradação e suporte a decisões de engenharia.

\subsection{Objetivos Específicos}
\begin{itemize}
    \item Integrar os quatro subconjuntos do C-MAPSS (FD001–FD004) em uma única base analítica, identificando o cenário de teste como uma dimensão de análise.
    \item Calcular métricas de confiabilidade e desempenho, incluindo Vida Útil Remanescente (RUL) e taxas médias de degradação.
    \item Modelar um DataMart com foco em engenharia, contendo dimensões como Motor, Cenário de Teste, Sensor, Tempo e Condição Operacional.
    \item Implementar consultas analíticas e dashboards para apoiar decisões de engenharia, tais como identificação de sensores críticos, comparação entre regimes operacionais e análise de tendências de falha.
    \item Demonstrar como o BI pode ser integrado a fluxos de PHM (\textit{Prognostics and Health Management}) para aumentar a visibilidade sobre o comportamento dos motores.
\end{itemize}

\section{Perguntas-Chave para o BI}

% --- PERGUNTAS (FOCO ENGENHARIA DE CONFIABILIDADE) ---
O sucesso da plataforma de BI será medido pela sua capacidade de responder a perguntas críticas de engenharia:
\begin{itemize}
    \item Quais sensores apresentam maior sensibilidade à degradação ao longo dos ciclos de operação? (\textit{Ajuda a identificar quais variáveis são mais relevantes para diagnósticos de saúde do motor.})
    \item Como a taxa média de degradação (queda de RUL) varia entre diferentes cenários de teste (FD001–FD004)? (\textit{Permite comparar o impacto das condições operacionais sobre a confiabilidade.})
    \item Quais motores demonstram comportamento anômalo em relação à média do grupo? (\textit{Detecta possíveis desvios experimentais ou diferenças no padrão de desgaste.})
    \item Quais parâmetros operacionais (altitude, Mach, ângulo de manete) mais influenciam a redução da vida útil? (\textit{Gera insights sobre o impacto das condições de voo na durabilidade.})
    \item Em média, qual é o ciclo de falha esperado para cada tipo de cenário? (\textit{Estabelece benchmarks internos de confiabilidade.})
    \item É possível identificar correlação entre sensores específicos antes da falha? (\textit{Permite mapear interdependência entre subsistemas (compressor, turbina, etc.).})
    \item Como o comportamento temporal dos sensores evolui nos últimos ciclos antes da falha? (\textit{Suporte direto a análises de degradação progressiva.})
\end{itemize}

% CAPÍTULO 2: METODOLOGIA E DESENVOLVIMENTO
\chapter{Metodologia e Desenvolvimento}

A construção da plataforma de Business Intelligence (BI) proposta neste trabalho seguirá as etapas clássicas de um projeto de Data Warehouse, conforme as abordagens de Kimball e Inmon, adaptadas ao contexto de engenharia de confiabilidade e ao domínio do conjunto de dados C-MAPSS. A metodologia compreende desde a compreensão do domínio e estrutura dos dados até sua modelagem dimensional e integração em um ambiente analítico, possibilitando o monitoramento da saúde de frota de motores turbofan simulados.

\section{Dataset NASA C-MAPSS}

O conjunto de dados C-MAPSS (Commercial Modular Aero-Propulsion System Simulation), desenvolvido pela NASA, é amplamente reconhecido como referência para pesquisas em Prognóstico e Gerenciamento de Saúde (PHM — \textit{Prognostics and Health Management}). Ele foi gerado por um modelo de simulação de alta fidelidade de motores turbofan comerciais, conforme descrito em \cite{saxena2008damage}.

O modelo considera as principais dinâmicas termodinâmicas e mecânicas do motor, permitindo simular o comportamento de seus componentes sob diferentes condições de operação e modos de falha. Esses dados possibilitam o desenvolvimento e teste de algoritmos de previsão de falhas, estimativa de vida útil restante (RUL) e suporte à manutenção preditiva.

\subsection{Modelo de Simulação}

\FloatBarrier

\begin{figure}[htbp]
    \centering
    \includegraphics[width=10cm]{images/turbofan_model.png}
    \caption{Diagrama simplificado do motor turbofan utilizado na simulação.}
    \label{fig:turbofan}
\end{figure}

\FloatBarrier

O modelo representado na Figura \ref{fig:turbofan} simula um motor turbofan de dois eixos (N1 e N2), composto por fan, compressores, câmaras de combustão e turbinas, incluindo sensores distribuídos ao longo de suas seções críticas. O processo de degradação dos componentes é modelado por meio de parâmetros de desempenho degradantes, como eficiência e vazão, que variam ao longo dos ciclos simulados.

\subsection{Fundamentos Termodinâmicos e Condições Operacionais}

O motor turbofan, assim como outros sistemas de propulsão a jato, opera segundo o \textbf{Ciclo de Brayton}, também conhecido como turbina a gás. Esse ciclo termodinâmico descreve o processo de conversão de energia química em energia mecânica e, posteriormente, em empuxo. Ele é composto por quatro etapas principais (Figura \ref{fig:brayton}):

\begin{enumerate}
\item \textbf{Compressão (adiabático):} O ar atmosférico é comprimido pelo fan e pelos compressores de baixa (LPC) e alta pressão (HPC), elevando sua pressão e temperatura.
\item \textbf{Combustão (isobárico):} O combustível é injetado e queimado na câmara de combustão, fornecendo energia térmica ao fluxo de ar comprimido.
\item \textbf{Expansão (adiabático):} Os gases quentes resultantes passam pelas turbinas de alta (HPT) e baixa pressão (LPT), que extraem energia para acionar os compressores e o fan.
\item \textbf{Exaustão (isobárico):} O ar expandido é expelido a alta velocidade, gerando empuxo conforme a Terceira Lei de Newton.
\end{enumerate}

\FloatBarrier

\begin{figure}[htbp]
    \centering
    \includegraphics[width=\textwidth]{images/brayton.png}
    \caption{Ciclo de Brayton (turbina a gás).}
    \label{fig:brayton}
\end{figure}

\FloatBarrier

A eficiência e o desempenho desse ciclo são fortemente influenciados pelas \textbf{condições operacionais externas}, representadas nas variáveis de configuração do dataset C-MAPSS — \textbf{altitude}, \textbf{número de Mach} e \textbf{posição da manete de potência}.

Essas variáveis refletem o regime de voo e determinam o ponto de operação do motor:

\begin{itemize}
    \item \textbf{Altitude:} Está relacionada à densidade do ar. Em maiores altitudes, a densidade atmosférica diminui, reduzindo a massa de ar admitida pelo fan e, consequentemente, a eficiência do ciclo.

    \begin{itemize}
        \item \textbf{O Modelo de Atmosfera Padrão (ISA - \textit{International Standard Atmosphere}):} Modelo utilizado para padronizar os cálculos de performance de aeronaves e motores. Este modelo matemático define como as propriedades do ar — pressão, temperatura e densidade — variam com a altitude. Assim, a altitude de operação não é apenas um valor geométrico, mas uma entrada direta para determinar o estado termodinâmico do ar que o motor irá admitir.
    \end{itemize}

    \item \textbf{Número de Mach ($Ma$):} Define a razão entre a velocidade da aeronave ($V$) e a velocidade local do som ($a$), conforme $Ma=V/a$. A velocidade local do som, por sua vez, não é constante; ela depende do estado termodinâmico do ar, especificamente de sua temperatura absoluta ($T$), sendo calculada por:
    $$a = \sqrt{\gamma R T}$$
    onde $\gamma$ é a razão dos calores específicos (adiabática, ~1,4 para o ar) e $R$ é a constante específica do gás (287 J/kg·K para o ar seco). O número de Mach está diretamente ligado à compressão dinâmica do ar na entrada do motor e à eficiência aerodinâmica das pás.

    \item \textbf{Posição da Manete de Potência (\textit{Throttle Resolver Angle} — TRA):} Representa o comando do piloto sobre a potência de empuxo. Um aumento no ângulo de manete eleva a vazão mássica de combustível ($W_f$) e as temperaturas de operação, intensificando o desgaste dos componentes térmicos.
\end{itemize}

Dessa forma, as variáveis de configuração operacional não apenas contextualizam as leituras dos sensores, mas também influenciam diretamente o comportamento de degradação simulado nos diferentes cenários do C-MAPSS. Essa relação é essencial para compreender as variações observadas nos subconjuntos FD001–FD004 e justifica sua modelagem como dimensões analíticas no \textit{Data Mart} proposto.

\subsection{Estrutura e Cenários de Teste}

O conjunto de dados C-MAPSS é composto por quatro subconjuntos — \textbf{FD001}, \textbf{FD002}, \textbf{FD003} e \textbf{FD004} —, cada um simulando uma frota de motores turbofan sob diferentes condições operacionais e modos de falha. Essa segmentação foi concebida para permitir a avaliação de algoritmos de prognóstico em diferentes níveis de complexidade e representa a principal dimensão de análise adotada neste projeto.

\begin{itemize}
\item \textbf{FD001:} Um único modo de falha (degradação do compressor de alta pressão — HPC) sob uma condição operacional constante.
\item \textbf{FD002:} Um único modo de falha (HPC), porém sob múltiplas condições operacionais (seis regimes distintos de operação).
\item \textbf{FD003:} Dois modos de falha (HPC e fan) sob condição operacional constante.
\item \textbf{FD004:} Dois modos de falha (HPC e fan) sob múltiplas condições operacionais (seis regimes distintos).
\end{itemize}

Essa estrutura hierárquica torna o C-MAPSS um conjunto de dados ideal para estudos de confiabilidade e manutenção preditiva, pois permite analisar como o comportamento dos sensores e a degradação do motor variam conforme o tipo de falha e o ambiente de operação.

A integração dos quatro subconjuntos em um único \textit{Data Mart} possibilitará análises comparativas de alto valor, permitindo observar o impacto combinado das condições de operação e dos modos de falha na vida útil dos motores. Essa consolidação é essencial para o desenvolvimento de painéis de Business Intelligence voltados à engenharia de confiabilidade, viabilizando a identificação de padrões de degradação e tendências de desempenho.

\subsection{Estrutura dos Dados}

Cada linha do conjunto de dados representa um ciclo de operação de um motor e contém 26 colunas, classificadas em três categorias principais:

\begin{itemize}
\item \textbf{Identificação:} Inclui o ID do motor e o número do ciclo correspondente, que funcionam como chaves primárias para rastrear o histórico de cada unidade ao longo do tempo (ciclos de operação).
\item \textbf{Configurações Operacionais (3 variáveis):} Representam o regime de voo e o contexto operacional do motor — por exemplo, altitude, número de Mach e posição do manete de potência. Essas variáveis determinam o ponto de operação no qual o motor está sendo simulado.

\item \textbf{Leituras de Sensores (21 variáveis):} Capturam a resposta física e termodinâmica do motor durante cada ciclo de operação. Entre elas, destacam-se:
    \begin{itemize}
        \item \textbf{Temperaturas (T2, T24, T30, T50):} Medem as temperaturas totais em diferentes estágios do motor, conforme a numeração do ponto de medição padrão em sistemas de propulsão aeronáutica:
        \begin{itemize}
            \item \textbf{T2:} Temperatura na entrada do fan (após a admissão de ar).
            \item \textbf{T24:} Temperatura após o \textbf{LPC} — \textit{Low Pressure Compressor} (compressor de baixa pressão).
            \item \textbf{T30:} Temperatura após o \textbf{HPC} — \textit{High Pressure Compressor} (compressor de alta pressão).
            \item \textbf{T50:} Temperatura após o \textbf{LPT} — \textit{Low Pressure Turbine} (turbina de baixa pressão).
        \end{itemize}
        
        \item \textbf{Pressões (P2, P15, P30):} Representam pressões totais em diferentes seções:
        \begin{itemize}
            \item \textbf{P2:} Pressão total na entrada do fan.
            \item \textbf{P15:} Pressão na região de bypass (entre o fan e o compressor de baixa pressão).
            \item \textbf{P30:} Pressão total após o HPC (compressor de alta pressão).
        \end{itemize}
        
        \item \textbf{Velocidades de Rotação (Nf, Nc):}
        \begin{itemize}
            \item \textbf{Nf:} Velocidade de rotação do eixo do fan — eixo de baixa pressão.
            \item \textbf{Nc:} Velocidade de rotação do eixo do núcleo (\textbf{core}) — eixo de alta pressão.
        \end{itemize}
        
        \item \textbf{Outros Indicadores:} Incluem variáveis como:
        \begin{itemize}
            \item \textbf{Wf:} Vazão mássica de combustível (\textit{Fuel Flow}).
            \item \textbf{EPR:} Razão de pressão do motor (\textit{Engine Pressure Ratio}), definida como a razão entre a pressão total na saída e na entrada do motor.
            \item Razões de fluxo e parâmetros derivados de eficiência, utilizados para monitoramento de desempenho.
        \end{itemize}
    \end{itemize}
\end{itemize}

A análise combinada dessas medições ao longo dos ciclos permite identificar padrões sutis de degradação associados ao desgaste progressivo dos componentes do compressor e da turbina. Em particular, o acompanhamento das variáveis relacionadas ao \textbf{HPC} e ao \textbf{fan} é fundamental, já que ambos os componentes estão associados aos principais modos de falha simulados no C-MAPSS.

\subsection{Pré-processamento e Modelagem Dimensional}
% Texto original mantido, pois a seção 2.4 detalha o que foi feito.
Para viabilizar o uso dos dados na plataforma de BI, serão aplicadas etapas de preparação que incluem:

\begin{enumerate}
\item \textbf{Limpeza e Padronização:} Verificação de consistência e formatação das variáveis conforme seus tipos físicos e unidades de medida.
\item \textbf{Engenharia de Atributos:} Cálculo de métricas derivadas, como gradientes de temperatura e variações de pressão entre estágios.
\item \textbf{Modelagem Dimensional:} Construção de um \textit{Data Mart} temático voltado à análise de saúde de frota, com tabelas fato (leituras por ciclo) e dimensões (motor, cenário, modo de falha, sensor, tempo).
\item \textbf{Integração com a Camada de BI:} Conexão do modelo ao ambiente analítico, permitindo visualizações interativas e consultas exploratórias sobre os sensores e parâmetros operacionais.
\end{enumerate}

\section{Base de Dados Operacional (OLTP)}
\textit{(A fonte de dados para este projeto não é um sistema OLTP transacional, mas sim os arquivos estáticos \texttt{.txt} fornecidos pela NASA, que representam logs de simulação. A Camada Bronze do nosso Data Warehouse atua como a primeira persistência desses dados brutos.)}
\newpage

\section{Arquitetura de Dados e Stack de Tecnologias}
A construção da plataforma de BI segue uma arquitetura de dados moderna baseada no conceito \textbf{Medallion} (Bronze, Silver, Gold) - Figura \ref{fig:medallion}. Esta abordagem organiza o fluxo de tratamento dos dados em etapas claras de qualidade e transformação, garantindo que os dados consumidos na camada final de BI sejam limpos, integrados e confiáveis.

\vspace{1cm}

\FloatBarrier

\begin{figure}[htbp]
    \centering
    \includegraphics[width=\textwidth]{images/medallion_architecture.png}
    \caption{Arquitetura Medalhão.}
    \label{fig:medallion}
\end{figure}

\FloatBarrier

\subsection{Stack de Tecnologias Utilizada}
A implementação desta arquitetura foi realizada utilizando a seguinte stack de tecnologias:
\begin{itemize}
    \item \textbf{Linguagem de Programação (ETL):} Python 3, com uso da biblioteca \texttt{pandas} para manipulação, transformação e processamento dos dados.
    \item \textbf{Banco de Dados (Data Warehouse):} MySQL, um sistema de gerenciamento de banco de dados relacional (SGBDR) objeto-relacional, escolhido para armazenar as três camadas do Data Warehouse (Bronze, Silver e Gold).
    \item \textbf{Ferramenta de BI (Consumo):} O destino final dos dados (Camada Gold) é ser consumido por uma ferramenta de visualização como Microsoft Power BI, onde os dashboards de engenharia serão construídos.
\end{itemize}

\subsection{Modelo Lógico e Dimensional (DataMart)}
O objetivo final é a \textbf{Camada Gold}, que funcionará como o DataMart dimensional. Esta camada será modelada em um esquema \textit{Star Schema} (ou floco de neve, se necessário), otimizado para consultas analíticas de BI. Ela será composta por:
\begin{itemize}
    \item \textbf{Tabela Fato:} Uma tabela central (ex: \texttt{Fato\_Leituras}) contendo as métricas principais (leituras de sensores, RUL) e as chaves estrangeiras para as dimensões.
    \item \textbf{Tabelas de Dimensão:} Tabelas que fornecem o contexto para a análise (ex: \texttt{Dim\_Motor}, \texttt{Dim\_Cenario\_Teste}, \texttt{Dim\_Tempo}, \texttt{Dim\_Condicao\_Operacional}).
\end{itemize}
O trabalho realizado até o momento (detalhado a seguir) focou na construção das camadas Bronze e Silver, que são pré-requisitos para a criação deste DataMart.
\newpage

\section{Processo de Extração, Transformação e Carga (ETL)}
O processo de ETL é o núcleo da engenharia de dados deste projeto, responsável por mover e tratar os dados desde a fonte bruta até o DataMart analítico.

\subsection{Trabalho Realizado (Camadas Bronze e Silver)}
O progresso atual do projeto compreende a extração e a transformação inicial dos dados, fundamentais para a criação do DataMart.

\subsubsection{Camada Bronze: Ingestão de Dados Brutos}
A primeira etapa, correspondente à \textbf{Camada Bronze}, consistiu na extração dos dados brutos dos quatro arquivos \texttt{train\_FD00X.txt} do dataset C-MAPSS. 
Um script em Python (\texttt{extracao.py}) foi desenvolvido para:
\begin{itemize}
    \item Ler os arquivos de texto, que são formatados com espaços e não possuem cabeçalho.
    \item Atribuir os nomes corretos às 26 colunas (ID do motor, ciclo, 3 configurações operacionais e 21 sensores).
    \item Adicionar uma coluna \texttt{dataset\_source} para identificar a origem de cada registro (ex: 'FD001'), o que é essencial para a futura \texttt{Dim\_Cenario\_Teste}.
    \item Carregar esses dados brutos, sem nenhuma outra transformação, em uma tabela de \textit{staging} no MySQL, cujo schema foi definido no script \texttt{cria\_schema.sql}.
\end{itemize}

\subsubsection{Camada Silver: Limpeza e Enriquecimento}
A \textbf{Camada Silver} representa a fonte de verdade limpa, validada e integrada. Nela, os dados da camada Bronze são transformados, enriquecidos e preparados para a modelagem analítica. O trabalho realizado nesta etapa (implementado em \texttt{silver\_tratamento.py} e \texttt{silver\_script.sql}) incluiu:
\begin{itemize}
    \item \textbf{Cálculo do RUL:} Conforme a hipótese do projeto, o KPI central (RUL) foi calculado para os dados de treino. Isso envolveu agrupar os dados por motor para encontrar o ciclo máximo (ciclo de falha) e, em seguida, calcular o RUL para cada linha como \texttt{(Ciclo\_Maximo - Ciclo\_Atual)}.
    \item \textbf{Padronização e Limpeza:} Verificação de tipos de dados e remoção de colunas que não agregam valor analítico (ex: sensores com leitura constante em todos os datasets).
    \item \textbf{Integração:} Consolidação dos quatro datasets (agora enriquecidos com RUL) em uma única tabela analítica na camada Silver, pronta para ser modelada na camada Gold.
\end{itemize}


\subsection{Definição e Hipótese do KPI Central: o RUL}
\subsubsection{O que é o RUL?}
A Vida Útil Remanescente, ou RUL (\textit{Remaining Useful Life}), é o KPI central deste projeto. Ele representa uma estimativa de quantos ciclos de operação um motor ainda possui antes de atingir um ponto de falha - podendo servir como parâmetro para identificação de um estágio seguro de degradação para a manutenção.

\subsubsection{A Hipótese do Projeto: BI como Consumidor de ML}
Na prática industrial, o RUL de um motor em operação é um valor desconhecido que precisa ser previsto. Esse prognóstico é uma tarefa clássica de Ciência de Dados e Machine Learning (ML), onde um modelo é treinado em dados históricos (como os do C-MAPSS) para aprender os padrões de degradação dos sensores que antecedem uma falha.

Para os fins deste projeto de DataMart e Business Intelligence, a premissa estabelecida é que este modelo de ML já existe e gerou a estimativa para as diferentes condições dos motores.

Nossa plataforma de BI não é responsável por criar a previsão. Ela atua na etapa seguinte: ela consome o resultado do modelo (o \texttt{RUL\_Previsto}) como seu principal KPI. O foco do nosso DataMart e dashboard é, portanto:
\begin{itemize}
    \item \textbf{Integrar:} Receber e armazenar o \texttt{RUL\_Previsto} para cada motor em cada ciclo.
    \item \textbf{Contextualizar:} Cruzar o \texttt{RUL\_Previsto} com as dimensões de engenharia (Cenário de Teste, Condições Operacionais).
    \item \textbf{Analisar:} Permitir que um engenheiro de confiabilidade monitore a queda do RUL e investigue os dados dos sensores que justificam aquela previsão.
\end{itemize}

Para simular o recebimento deste KPI no nosso processo de ETL, usaremos os dados de treino do C-MAPSS (onde a falha é conhecida) para calcular o RUL verdadeiro através de aritmética simples (ex: \texttt{RUL = Ciclo\_Maximo - Ciclo\_Atual}). Este RUL calculado na Camada Silver servirá como o \texttt{RUL\_Previsto} em nosso DataMart.

\newpage

\section{Consultas Analíticas}
\textit{(Espaço reservado para a Entrega 6: Apresentação e explicação dos 5 SELECTs com funções analíticas, focados em responder às perguntas de engenharia.)}
\newpage

% CAPÍTULO 3: RESULTADOS E ANÁLISE 
\chapter[Resultados e Análise]{Resultados e Análise (Dashboard)}
\textit{(Espaço reservado para a Entrega 7: Apresentação dos 5 gráficos/dashboard e a análise dos insights de engenharia.)}

% CONCLUSÃO
\chapter*{Conclusão}
\phantomsection
\addcontentsline{toc}{chapter}{Conclusão}

Este projeto propõe uma arquitetura de Business Intelligence robusta, focada em atender às necessidades da Engenharia de Confiabilidade. Ao transformar dados de simulação de alta fidelidade em uma plataforma analítica, espera-se demonstrar o valor do BI como ferramenta de diagnóstico e suporte à decisão no ciclo de vida de componentes complexos, como motores aeronáuticos.

% --- ELEMENTOS PÓS-TEXTUAIS ---
\postextual

% --- CONFIGURAÇÕES DO BIBTEX ---
\bibliography{referencias} 
\bibliographystyle{abntex2-num} 

\end{document}
